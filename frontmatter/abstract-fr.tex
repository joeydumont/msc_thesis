% !TeX root = ../msc_thesis_jayd.tex
\chapter*{Résumé}
\phantomsection\addcontentsline{toc}{section}{Résumé}
Les percées technologies des dernières années ont changé
la vision que nous avons de la science et ont même changé
la façon de faire de la science. Il fut un temps, un temps
bien avant le baptême de la physique, où la science était
en grande partie une activité expérimentale. Allant 
contre les préjugés de ses contemporains, Werner
Heisenberg su établir la physique théorique comme une
discipline en soi. Depuis, néanmoins, il semble que la
discipline ait perdue de son éclat; la facilité
avec laquelle nous pouvons générer des données semble avoir
engendrée l'idée qu'elles sont autosuffisantes, 
qu'elles fournissent elles-mêmes la raison de leur existence, 
que le trône de la théorie a été usurpé par l'opulence
des données [\textit{WIREs Comput. Stat.} \textbf{6}(2), 75--79 (2014)]. 
Dans ce travail, nous tentons de renverser
la tendance à marginaliser l'importance d'un modèle 
théorique en montrant que l'analyse mathématique peut donner une 
compréhension physique intrinsèquement inatteignable 
aux données. 

Premièrement, nous allons explorer la modélisation des
cavités diélectriques bidimensionnelles. La dernière
décennie a montré que ce type de dispositifs, dont la 
taille est de l'ordre du micromètre, est riche en applications, 
allant de la biodétection jusqu'aux lasers. Ce large éventail de
possibilités a motivé la communité scientifique à étudier
le comportement de ces structures dans différentes situations.
La plupart des études, autant théoriques qu'expérimentales, 
se concentrent sur l'effet d'une modification de la géométrie 
sur la réponse de la cavité et considèrent un 
indice de réfraction constant à l'intérieur de la cavité. 
Dans ce mémoire, nous allons développer différentes méthodes
de modélisation valides pour des cavités diélectriques à géométrie
arbitraires et à profil d'indice de réfraction arbitraire. Ce
degré de liberté supplémentaire pourra être utilisé comme variable
supplémentaire dans le design de microcavités pour des applications
spécifiques. 
Un formalisme de diffusion permettra de définir les modes caractéristiques
de ce type de structure et d'en calculer les résonances. Une analyse
numérique des équations résultantes montrera que les méthodes intégrales
sont possiblement meilleures que les méthodes différentielles.

Deuxièmement, nous discuterons de la modélisation de structures
radiatives. Nous utiliserons les méthodes développées dans la section
précédente pour modéliser les propriétés lasers des microcavités
bidimensionnelles prédites par la théorie SALT. Nous aborderons
aussi la modélisation de fibres-antennes RF, plus particulièrement
les câbles coaxiaux à perte radiative, dans le but d'intégrer
des fonctionnalités radio dans un textile de manière transparente
à l'utilisateur. 