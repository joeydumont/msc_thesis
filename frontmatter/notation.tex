% !TeX root = ../msc_thesis_jayd.tex
\chapter*{Notation}
\phantomsection\addcontentsline{toc}{section}{Notation}

% Model
% \begin{tabularx}{\textwidth}{lX}
%   \hline\hline
%   Symbol		& Definition	\\
%   \hline\hline				
%   \hline\hline				\\
% \end{tabularx}

\section*{Mathematical Operators}
\begin{tabularx}{\textwidth}{lX}
  \hline\hline
  Symbol				& Definition	\\
  \hline\hline				
  $\displaystyle{\bigk_{m=1}^\infty\left(\frac{a_m}{b_m}\right)}$
						& Continued fraction expansion with coefficients $a_m$ and $b_m$. \\
  ${\sum_m}$			& Sum over all angular momenta in two dimensions, $m=-\infty\ldots\infty$.\\
  ${\sum_{\ell,m}}$		& Sum over all angular momenta in three dimensions, $\ell=0\ldots\infty$, $m=-\ell\ldots\ell$. \\
  ${\sum_{\sigma,\ell,m}}$	& Sum over all spinorial angular momenta in three dimensions.	\\
  $\floor*{x}$			& Floor function. Maps $x$ to the largest previous integer.	\\
  $\ceil*{x}$			& Ceiling function. Maps $x$ to the smallest following integer.	\\
  $\avg{x}$				& Average value of quantity between brackets.			\\
  \hline\hline				\\
\end{tabularx}

\section*{Vectors and Matrices}
\begin{tabularx}{\textwidth}{lX}
  \hline\hline
  Symbol				& Definition	\\
  \hline\hline				
  $\bo{V}$, $\bo{v}$	& Vectors (bold and italic, uppercase and lowercase).	\\
  $V_i, v_i$			& $i$th component of vector.	\\
  $\Bra{\alpha}$, $\Ket{\beta}$ 
  						& Bra-ket notation. Roughly equivalent to row and column vectors, respectively.	\\ 
  $\mat{M}$				& Matrix (bold uppercase)	\\
  $M_{ij}$				& Element on $i$th row and $j$th column. $(j+iN)$th element of matrix (row-major ordering) 
			 				where $N$ is the number of columns. The indices start at 0.\\
  $S_{mm'}$, $Q_{mm'}$	& Element of matrices in the 2D angular momentum basis. Indices span $m,m'\in\left\{-M_\text{max}\ldots M_\text{max}\right\}$.	\\
  $g_{\mu\nu}$			& Metric tensor. Indices span only the spatial coordinates, i.e. no time coordinate. \\
  $[\alpha\beta,\gamma]$& Christoffel symbol of the first kind.	\\
  ${\Gamma^{\alpha}}_{\beta\gamma}$
						& Christoffel symbol of the second kind. \\
  \hline\hline				\\
\end{tabularx}

\section*{Electromagnetism}
\begin{tabularx}{\textwidth}{lX}
 \hline\hline
 Symbol								& Definition 	\\
 \hline\hline
 $\bo{E}$, $\bo{D}$, $\bo{P}$		& Electric field, electric displacement and polarization.	\\
 $\bo{H}$, $\bo{B}$, $\bo{M}$		& Magnetic field, magnetic induction and magnetization.	\\
 $\bo{j}$, $\rho$					& Current and charge densities.	\\
 $\bo{\chi}_{e,m}(\bo{r},\bo{r}',t,t')$	& Electric/magnetic tensorial susceptibility. 	\\
 $\epsilon=\epsilon'+i\epsilon''$	& Electric permittivity of a material. We often separate its real and imaginary parts as shown.	\\
 $\mu$								& Magnetic permeability of a material. \\
 $n=\sqrt{\epsilon\mu}$				& Refractive index of a material. 	\\
 $\sigma_e$							& Electric conductivity of a material. \\
 \hline\hline
\end{tabularx}


\section*{Special Functions}
\begin{tabularx}{\textwidth}{lX}
  \hline\hline
  Symbol			& Definition	\\
  \hline\hline
  $J_\nu(z)$		& Bessel function of the first kind of order $\nu$.	\\
  $Y_\nu(z)$		& Bessel function of the second kind of order $\nu$.	\\
  $H_\nu^{(\pm)}(z)$, %
  $H_\nu^{(\omega)}(z)$& Hankel functions of the first and second kind, respectively, of order $\nu$, $\omega=\pm$.\\
  $U(a,b,z)$		& Kummer's function, also known as the confluent hypergeometric function of the first kind, with parameters $a$, $b$. \\
  \hline\hline				\\
\end{tabularx}

\section*{Angular Momentum}
\begin{description}
 \item[Ordering of angular momentum in 2D.] We consider both positive and angular momentum, so the matrices have size $2M+1\times2M+1$ where
					    $M$ is the maximum angular momentum.
  \begin{center}
  \begin{tikzpicture}[>=latex]
   \node at (-4.5,0) {$\mat{V}\bo{\psi} = $};
   \matrix (A) [matrix of math nodes,%
		%nodes = {node style ge},%
		left delimiter=(,%
		right delimiter=) ] at (0,0)
  { \ddots				& \phantom{\ddots}	& \phantom{\ddots}	& \phantom{\ddots}	& \iddots	\\
    \phantom{\ddots}	& V_{-1-1}			& V_{-10}			& V_{-11} 			& \phantom{\ddots}	\\
    \phantom{\ddots}	& V_{0-1}			& V_{00}			& V_{01} 			& \phantom{\ddots}	\\
    \phantom{\ddots}	& V_{1-1}			& V_{10}			& V_{11}			& \phantom{\ddots}	\\
    \iddots				& \phantom{\ddots}	& \phantom{\ddots}	& \phantom{\ddots} 	& \ddots	\\
  };
  
  \draw [thick,decorate,<->,>=stealth] ($(A-1-1.north west)+(0,\myunit)$) -- ($(A-1-5.north east)+(0,2*\myunit)$)
	node[above=2.0*\myunit,midway] {$m'$}
	node[above=0.5*\myunit,midway] {$0$}
	node[above=0.5*\myunit,very near end] {$+$}
	node[above=0.5*\myunit,very near start] {$-$};
  \draw [thick,decorate,<->,>=stealth] ($(A-1-1.north west)-(2.25*\myunit,0)$) -- ($(A-5-1.south west)-(2.40*\myunit,0)$)
	node[left=2.0*\myunit,midway] {$m$}
	node[left=0.5*\myunit,midway] {$0$}
	node[left=0.5*\myunit,very near end] {$+$}
	node[left=0.5*\myunit,very near start] {$-$};
  
   \matrix (B) [matrix of math nodes,%
		%nodes = {node style ge},%
		left delimiter=(,%
		right delimiter=),%
		right=20pt] at (A.east)
  { \vdots \\ \psi_{-1} \\ \psi_{0} \\ \psi_{1} \\ \vdots \\
  };
  \end{tikzpicture}
  \end{center}

\item[Ordering of angular momenta $\ell$, $m$ in 3D.] This block structure allows the product 
						      \begin{equation*}\sum_{\ell'=0}^\infty\sum_{m'=-\ell'}^{\ell'} V_{\ell m,\ell' m'}\psi_{\ell'm'}\end{equation*} 
						      to be written using a single index $\nu = (\ell,m)$. The matrices have size 
						      $(\ell_\text{max}+1)^2\times(\ell_\text{max}+1)^2$.
  \begin{center}
  \begin{tikzpicture}[>=latex]
   \node at (-6,0) {$\mat{V}\bo{\psi} = $};
   \matrix (A) [matrix of math nodes,%
		%nodes = {node style ge},%
		left delimiter=(,%
		right delimiter=) ] at (0,0)
  { V_{\color{blue}{00},\color{red}{00}}	& V_{\color{blue}{00},\color{red}{1-1}}	& V_{\color{blue}{00},\color{red}{10}}	& V_{\color{blue}{00},\color{red}{11}}	\\
    V_{\color{blue}{1-1},\color{red}{00}}	& V_{\color{blue}{1-1},\color{red}{1-1}}& \vdots								& \vdots	\\
    V_{\color{blue}{10},\color{red}{00}}	& \ldots								& \ddots								& \vdots	\\
    V_{\color{blue}{11},\color{red}{00}}	& \ldots								& \ldots								& \ddots	\\
  };
  
  \draw [thick,decorate,decoration={brace,amplitude=5pt}] (A-1-1.north west) -- (A-1-1.north east) node [midway,above=\myunit] {\color{red}{$\ell'=0$}};
  \draw [thick,decorate,decoration={brace,amplitude=5pt}] (A-1-2.north west) -- (A-1-4.north east) node [midway,above=\myunit] {\color{red}{$m'=-\ell',\ldots,\ell'$}};
  \draw [thick,decorate,decoration={brace,mirror,amplitude=5pt}] ($(A-1-1.north west)-(1cm,0)$) -- ($(A-1-1.south west)-(1cm,0)$) node [midway,left=\myunit] {\color{blue}{$\ell=0$}};
  \draw [thick,decorate,decoration={brace,mirror,amplitude=5pt}] ($(A-2-1.north west)-(0.9cm,0)$) -- ($(A-4-1.south west)-(1cm,0)$) node [midway,left=\myunit] {\color{blue}{$\ell=1$}};
  
   \matrix (B) [matrix of math nodes,%
		%nodes = {node style ge},%
		left delimiter=(,%
		right delimiter=),%
		right=20pt] at (A.east)
  { \psi_{00} \\ \psi_{1-1} \\ \psi_{10} \\ \psi_{11} \\ \vdots \\
  };
  \end{tikzpicture}
  \end{center}
\end{description}

