\chapter*{Abstract}
\phantomsection\addcontentsline{toc}{section}{Abstract}

In recent years, outbursts in technology have changed the way 
we view, and do, science. In the days of yore, when the rose
that is physics had not yet been given its name, science
was in major part an experimental endeavour. It took the great
Werner Heisenberg to establish theoretical physics as its own field. 
Today, however, it seems that the incredible ease with which we 
can produce data, throves upon throves of it, has spawned the idea
that data is law, that data is self-suficient, that the throne of 
theory has been usurped by the opulent and almight data. In this
thesis, we seek to reverse this trend by showing that theoretical 
analysis can yield physical insights inherently inaccessible to
raw data. 

We will first investigate the modelization of bidimensional dielectric
cavities. The last decade has shown that this type of micron-sized device
is ripe with applications ranging from biodetection to lasing action (more on that
later). This has motivated the scientific community to investigate the behaviour
of such cavities mostly as a function of their geometry, leaving their refractive index
profile constant. We will generalize a scattering method that can deal with 
arbitrary geometry and arbitrary refractive index profiles and provides
a way to accurately compute the resonances of such structures. 
A numerical analysis of othe underlying equations of the theory will
reveal that perhaps it is best to forego differential equations 
in favour of integral ones for the scattering problem. 

In the second part, we will discuss the modelization of radiating structures. 
Using the formalism developed in the previous section, we will study in detail the 
lasing properties of bidimensional cavities using the newly developed
\textit{self-consistent} ab initio \textit{laser theory} (SALT). We will
also attemt to model a class of antenna known as \textit{leaky coax} antennas using
a mix of semi-analytic and all-numerical methods. More specifically, we will attempt
to explain the gap between the experimental data and simulation data. 