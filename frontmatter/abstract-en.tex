\chapter*{Abstract}
\phantomsection\addcontentsline{toc}{section}{Abstract}

In recent years, outbursts in technology have changed the way 
we view, and do, science. In the days of yore, when the rose
that is physics had not yet been given its name, science
was in major part an experimental endeavour. It took the great
Werner Heisenberg to establish theoretical physics as its own field. 
Today, however, it seems that the incredible ease with which we 
can produce data, troves upon troves of it, has spawned the idea
that data is law, that data is self-suficient, that the throne of 
theory has been usurped by the opulent and almighty data. In this
thesis, we seek to reverse this trend by showing that theoretical 
analysis can yield physical insights inherently inaccessible to
raw data. 

In the first part, we will first investigate the modelization of bidimensional dielectric
cavities. The last decade has shown that this type of micron-sized device
is ripe with applications ranging from biodetection to lasing action. 
This has motivated the scientific community to investigate the behaviour in a
variety of different situations. Most studies concern themselves with the effect
of the geometry of the cavity on its response, leaving their refractive index
profile constant. In this essay, we will develop different modelization techniques valid 
for cavities having arbitrary geometries and refractive index profiles and provide
a way to accurately compute the resonances of such structures. The refractive
index thus becomes an additional design variable for dielectric cavities.
A numerical analysis of othe underlying equations of the theory will
reveal that perhaps it is best to forego differential equations 
in favour of integral ones for the scattering problem. 

In the second part, we will discuss the modelization of radiating structures. 
Using the formalism developed in the previous section, we will study in detail the 
lasing properties of bidimensional cavities using the newly developed
\textit{self-consistent} ab initio \textit{laser theory} (SALT). We will also
touch on the modelization of the class of antenna known as \textit{leaky coax} antennas, 
whose purpose are to be seamlessly integrate radio functionality onto textile.