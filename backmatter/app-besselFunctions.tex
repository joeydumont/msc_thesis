% !TeX root = ../msc_thesis_jayd.tex
\chapter{Bessel Functions}
This appendix contains some results concerning the 
Bessel functions. Given our redution of Maxwell's equations 
from 3D to 2D using cylindrical coordinates, the 
Bessel functions will form the basis of our analysis, 
as they are the eigenfunctions of
Helmholtz's equation in those coordinates.

We thus wish to collect some of their most important
properties here for easy reference. Most of them come from the celebrated
volume by Abromowitz \& Stegun \cite{ABR1965}, while a few
come from \cite{CUY2008}. Proper reference will be given when needed. 

In what follows, $\nu,z\in\mathbb{C}$ and $n\in\mathbb{N}$ unless explicitly
stated otherwise.

\section{Definition and Elementary Properties}

\subsection{Differential Equation}
The Bessel functions solve the differential equation
  \begin{equation}
   \label{eq:app.Bessel.diffEquation}
   z^2\frac{d^2w}{dz^2}+z\frac{dw}{dz}+(z^2-\nu^2)w=0
  \end{equation}
which is a special case of the confluent hypergeometric
differential equation, which is in turn a special case
of the hypergeometric differential equation. When solved
via Fröbenius' method, it yields the solution
  \begin{equation}
   \label{eq:app.Bessel.seriesJ}
   J_\nu(z) = \sum_{k=0}^\infty \frac{(-1)^k\left(\frac{1}{2}z\right)^{\nu+2k}}{k!\Gamma(\nu+k+1)}.
  \end{equation}
This is the Bessel function of the first kind. A second, linearly independent
is defined by
  \begin{equation}
   Y_\nu(z) = \frac{J_\nu(z)\cos\pi\nu-J_{-\nu}(z)}{\sin\pi\nu}
  \end{equation}
where a valid limiting process must be used when $\nu\rightarrow n$. 

We will be mostly interested in a second set of linearly independant
solutions: the Hankel functions. They are defined by 
  \begin{subequations}
  \label{eq:app.Bessel.hankelDef}
  \begin{align}
   H_\nu^{(+)}(z)	&= J_\nu(z)+iY_\nu(z)	\\
   H_\nu^{(-)}(z)	&= J_\nu(z)-iY_\nu(z).
  \end{align}
  \end{subequations}
We will alternatively use the notation
  \begin{equation}
   H_\nu^{(\omega)} = J_\nu(z)+i\omega Y_\nu(z)\qquad (\omega=\pm)
  \end{equation}
to denote both functions at the same time.

\subsection{Recurrence Relations}
From the differential equation itself, we can derive multiple
recurrence relations that the whole family of Bessel functions 
obey. If $\mathcal{C}$ denotes $J$, $Y$, $H^{(\pm)}$ or any 
linear combination of the four functions, we have
  \begin{subequations}
   \label{eq:app.Bessel.recurrence}
  \begin{align}
   \mathcal{C}_{\nu-1}(z)+\mathcal{C}_{\nu+1}(z)		&= \frac{2\nu}{z}\mathcal{C}_\nu(z)	\label{eq:app.Bessel.recurrenceBessel}\\
   \mathcal{C}_{\nu-1}(z)-\mathcal{C}_{\nu+1}(z)		&= 2\mathcal{C}_\nu'(z)			\label{eq:app.Bessel.recurrenceDiffBessel}\\
   \mathcal{C}_{\nu-1}(z)-\frac{\nu}{z}\mathcal{C}_\nu(z)	&= \mathcal{C}_\nu'(z)			\\
   -\mathcal{C}_{\nu+1}(z)+\frac{\nu}{z}\mathcal{C}_\nu(z)	&= \mathcal{C}_\nu'(z).
  \end{align}
  \end{subequations}
However, the minimal solution to these recurrence relations is 
$J_\nu(z)$, so any attempt at numerically evaluating $Y_\nu$ or $H_\nu^{(\pm)}$
using these is foiled (see \S\ref{sec:app.numTools.logDeriv} for details). 

\subsection{Relations between Solutions}
The following are analytical relationships between the set of 
Bessel functions. They can be of use in both standard and numerical 
analysis. 

\begin{description}
 \item Reflection Formulas \cite[p.~286]{PRE2007}
  \begin{subequations}
  \begin{align}
   J_{-\nu}(z)	&= \cos\nu\pi J_\nu(z)-\sin\nu\pi Y_\nu(z)	\\
   Y_{-\nu}(z)	&= \sin\nu\pi J_\nu(z)+\cos\nu\pi Y_\nu(z)	\\
   J_{-n}(z)	&= (-1)^nJ_n(z)					\\
   Y_{-n}(z)	&= (-1)^nY_n(z)					\\
   H_{-\nu}^{(\omega)} &= e^{i\omega\nu\pi}H_\nu^{(\omega)}
  \end{align}
  \end{subequations}
 \item Complex Conjugate ($\nu\in\mathbb{R}$)
  \begin{subequations}
  \begin{align}
   \overline{J_\nu(z)}	&= J_\nu(\overline{z})	\\
   \overline{Y_\nu(z)}	&= Y_\nu(\overline{z})	\\
   \overline{H_\nu^{\omega}(z)} &= H_\nu^{(-\omega)}(\overline{z})\label{eq:app.Bessel.conjHankel}
  \end{align}
  \end{subequations}
\end{description}

The last of these equations is of particular importance in establishing
symmetries of the scaattering matrix.

\section{Asymptotic and Limiting Forms}
\subsection{Expansions for Small Arguments and Fixed $\nu$}\label{sec:app.Bessel.smallArguments}
From the first few terms of the power series \eqref{eq:app.Bessel.diffEquation}, 
  \begin{equation}
    J_\nu(z) = \frac{1}{\Gamma(\nu+1)}\left(\frac{z}{2}\right)^\nu\left[1-\frac{1}{\nu+1}\left(\frac{z}{2}\right)^2\right]+\mathcal{O}(z^{\nu+4}).
  \end{equation}
  
Using the first terms of the ascending series for integer orders $n$ \cite[\S9.1.11]{ABR1965}
  \begin{multline}
   Y_n(z)=-\frac{1}{\pi}\left(\frac{z}{2}\right)^{-n}\sum_{k=0}^{n-1}\frac{(n-k-1)!}{k!}\left(\frac{z}{2}\right)^{2k}
      +\frac{2}{\pi}\ln\frac{z}{2}J_n(z)
      \\-\frac{1}{\pi}\left(\frac{z}{2}\right)^n\sum_{k=0}^\infty\left\{\psi(k+1)+\psi(n+k+1)\right\}\left(\frac{z}{2}\right)^{2k}\frac{1}{k!(n+k)!},
  \end{multline}
where 
  \begin{equation}
   \psi(n+1) = -\gamma +\sum_{k=1}^nk^{-1}\qquad(\gamma=0.57721\,56649\ldots)
  \end{equation}
is the digamma function. The cases $n=0$ and $n\neq0$ differ
  \begin{align}
   Y_0(z)	&= \frac{2}{\pi}\left(\ln\frac{z}{2}+\gamma\right)+\mathcal{O}(z^2)	\\
   Y_n(z)	&= -\frac{(n-1)!}{\pi}\left(\frac{z}{2}\right)^{-n} +\mathcal{O}(z^{-n+2}).
  \end{align}

The expansions of the Hankel functions are found by using 
their definition \eqref{eq:app.Bessel.hankelDef}:
  \begin{align}
   H_0^{(\omega)}(z)	&=1+\frac{2i\omega}{\pi}\left(\ln\frac{z}{2}+\gamma\right)+\mathcal{O}(z^2)	\\
   H_n^{(\omega)}(z)	&=-\frac{i\omega(m-1)!}{\pi}\left(\frac{z}{2}\right)^{-n}+\mathcal{O}(z^{-n+2}).
  \end{align}

\subsection{Expansions for Large Arguments and Fixed $\nu$}
The Bessel functions, jealous of the simpler trigonometric functions, 
try to mimic them when their arguments get large. We have
  \begin{subequations}
  \begin{align}
   J_\nu(z)		&\approx \sqrt{\frac{2}{\pi z}}\cos\left(z-\frac{\nu\pi}{2}-\frac{\pi}{4}\right)	\\
   Y_\nu(z)		&\approx \sqrt{\frac{2}{\pi z}}\sin\left(z-\frac{\nu\pi}{2}-\frac{\pi}{4}\right)	\\
   H_\nu^{(\omega)}(z)	&\approx \sqrt{\frac{2}{\pi z}}e^{i\omega\left(z-\frac{\nu\pi}{2}-\frac{\pi}{4}\right)}\label{eq:app.Bessel.asymptoticHankel}
  \end{align}
  \end{subequations}
The last of these makes the respect of Sommerfeld's radiation condition a breeze in cylindrical coordinates.
