\chapter{Basic Equations of Scattering}\label{app:basicEquations}

\section{SALT Equations for TE and TM Polarizations}

\section{Specialization to Dielecric Cavities}

\section{Vector Scattering and the Variable Phase Method}

\section{Green's Function Methods}

\subsection{Scattering in 3D and Antennae}
Somewhat departing from the previous chapter, we here present
three different methods to compute the current distribution on 
an arbitrary antenna. First, we will state the usual result 
from \cite{ELL2003} and then we will take the scattering viewpoint.

From our previous
viewpoint that sources generate electric and magnetic fields 
and using Maxwell's equations outside the region where the sources 
are, we can obtain \cite{ELL2003}
  \begin{equation}
    \mathop{\iiint}_V \mat{K}(\bo{r},\bo{r}')\bo{J}(\bo{r}')d\bo{r}' = i\omega\bo{E}(\bo{r})
  \end{equation}
where the kernel $\mat{K}(\bo{r},\bo{r}')$ is 
  \begin{equation}
    \mat{K}(\bo{r},\bo{r}') = \left(\nabla^2+k^2\right)\frac{e^{ik|\bo{r}-\bo{r}'|}}{4\pi|\bo{r}-\bo{r}'|}
  \end{equation}
For antennae made of perfect conductors ($\sigma\rightarrow\infty$), we must have $\bo{E}\rightarrow0$
except in the feeding region. If $\bo{E}$ is known in the feeding region, this becomes an integral equation
for the unknown current distribution. Specifically, it becomes a vectorial Fredholm equation of the first kind. 
Its vectorial nature makes it more difficult to solve. 

It is unknown to the author of this report why Elliott insists on separating the electric field and the current. 
By using Ohm's law, we can arrive at integral equations which have a single unknown. This is the 
scattering viewpoint. 
  
Consider an antenna of arbitrary geometry and arbitrary
permittivity and conductivity but of  vacuum permeability embedded in an infinite
medium of constant permittivity $\epsilon_B(\omega)$. For non-magnetic materials
and harmonic time-dependence, 
the curl equation for the electric field is 
  \begin{align*}
   \nabla\times\nabla\times\bo{E}	&=ik\left[\nabla\times\bo{H}\right]	\\
					&=ik\left[-ik\epsilon_B(\omega)\bo{E}+\bo{J}\right]	\\
					&=k^2\left[\epsilon_B(\omega)\bo{E}-\frac{\sigma}{ik}\bo{E}\right]	\\
					&=k^2\epsilon_B(\omega)\bo{E}
  \end{align*}
where we used Ohm's law
  \begin{equation}
   \bo{J}=\sigma\bo{E}
  \end{equation}
and redefined 
  \begin{equation}
   \epsilon_B(\omega)\mapsto\epsilon_B(\omega)+i\sigma/k
  \end{equation}

For our antenna, the curl equation for 
the electric field become
  \begin{subequations}
  \begin{align}
   \nabla\times\nabla\times\bo{E}	&= k^2\epsilon_B(\omega)\bo{E}		& \bo{r}\notin V_\text{scat}	\\
   \nabla\times\nabla\times\bo{E}	&= k^2\epsilon(\bo{r};\omega)\bo{E}	& \bo{r}\in V_\text{scat}
  \end{align}
  \end{subequations}
By subtracting $k^2\epsilon_B(\omega)\bo{E}$ from both sides of the second equation, 
we obtain an equation that is valid both inside and outside the antenna (or scatterer)
  \begin{align}
   \nabla\times\nabla\times\bo{E}-k^2\epsilon_B(\omega)\bo{E}	&= k^2\Delta\epsilon(\bo{r};\omega)\bo{E}	&\forall\bo{r} \label{eq:scattering.integral}
  \end{align}
where 
  \begin{equation}
    \Delta\epsilon(\bo{r};\omega)=\epsilon(\bo{r},\omega)-\epsilon_B(\omega).
  \end{equation}

The vector equation \eqref{eq:scattering.integral} has a formal solution in terms 
of a dyadic Green's function \cite{NOV2012} and an incident field \cite{deL2013}
  \begin{equation}
   \bo{E}(\bo{r}) = \bo{E}_\text{inc}(\bo{r}) + k^2\mathop{\iiint}_{V_\text{scat}} \mat{G}(\bo{r},\bo{r}')\Delta\epsilon(\bo{r}')\bo{E}(\bo{r}')d\bo{r}'\label{eq:scattering.Fredholm2nd}
  \end{equation}
where the Green's function is given by, in coordinate-free representation
  \begin{align}
    \mat{G}(\bo{r},\bo{r}')	&= \left(\mat{I}+\frac{1}{k\sqrt{\epsilon_B}}\nabla\nabla\right)g(\bo{r},\bo{r}')	\\
    g(\bo{r},\bo{r}')		&= \frac{\exp(i\sqrt{\epsilon_B}k|\bo{r}-\bo{r}'|)}{4\pi|\bo{r}-\bo{r}'|}.
  \end{align}
We note that this is a multidimensional, vector Fredholm integral equation of the second kind with
singular kernel $\mat{G}(\bo{r},\bo{r}')\Delta\epsilon(\bo{r}')$.

Let's take a moment to take in equation \eqref{eq:scattering.Fredholm2nd}. It is the general
solution to the scattering of light on \textit{any} non-magnetic material. The material can be
inhomogeneous, anisotropic, of arbitrary geometry and even non-linear. Moreover, it considers
the complete vector field: we are not forced to set TM/TE polarization. This generality, however, 
comes at a cost. While uni-dimensional Fredholm equation of the second type are usually easily solvable
(even when the kernel is singular) \cite{DEL1985}, the multidimensional case is intractable unless
we use basis function expansion to factor out the angular dependence of the 
field and kernel. This is what is done in \cite{deL2013} in the case of linear, isotropic, 
spherical scatterers.

The situation  is 
further complicated by the fact that the kernel of the integral equation is singular. 
One way to remedy this problem is to embrace the singularity. Given the form
of the Green dyadic, a way to construct an integration routine would be
to notice that, if $r$ and $r'$ are the amplitudes of the vectors
  \begin{equation}
   \frac{1}{|\bo{r}-\bo{r}'|} = \frac{1}{r_>}\sum_{\ell=0}^\infty \left(\frac{r_<}{r_>}\right)^\ell P_\ell(\cos\gamma)
  \end{equation}
where $\gamma$ is the angle between $\bo{r}$ and $\bo{r}'$, $r_<=\text{min}(r,r')$, 
$r_>=\text{max}(r,r')$. 
This allows us to factor out the singular part of the kernel. There will 
then be poles of orders up to $\ell+1$ which we can use
as a weight for an integration routine.

Experience tells us that multidimensional integration is never easy and 
computationally costly. We will steer away from this method of solution
from the time being. Of course, by expanding the solution in 
vector spherical harmonics, we could factor the radial singularity
and construct an appropriate single dimensional integration routine. 
This will be the basis of our next, and adopted, solution.

\subsection{Effective Index Approximation}
In our last report, we derived general equations for the
behaviour of light in bidimensional structures, approximating 
the finite thickness resonator by an infinite thickness one. In 
what follows, we will rigorously take into account the finite thickness
of the bidimensional resonator. Having these exact equations might enable us
to study the bidimensional limit of the electromagnetic theory. 

Given the coordinate-free representation of the Green dyadic, we can solve
the scattering problem in cylindrical coordinates. Say  we have
a thin dielectric resonator of thickness $t_z$ whose boundary is given
by the function $R(\phi)$. The refractive
index inside the dielectric is given by $\epsilon=\epsilon(r,\phi)$ and
is constant outside. We have (if the functional dependence is omitted, it is implicitly
assumed to be on $\bo{r}$)
  \begin{equation}
   \bo{E} = \bo{E}_\text{inc}
	  +k^2\int_{0}^{R(\phi)}\int_0^{2\pi}\int_{-t_z/2}^{t_z/2}\mat{G}(r,\phi,z|r',\phi',z')\Delta\epsilon(r',\phi')\bo{E}(r',\phi',z')dz'd\phi'dr'
  \end{equation}
Perhaps the simplest way to retain the simplicity of the integrals is to expand the fields and the dyadic Green's function
in vector cylindrical harmonics. However, this expansion requires the evaluation of an integral \cite{BEN1968}, making 
for intensive numerical work. 

One other way to approach this would be to to evaluate the integrals in spherical coordinates. In that 
case, we could separate the angular behaviour from the radial behaviour and end up with 
a uni-dimensional Fredholm equation which we could solve by using either the resolvent formalism
or linear algebra techniques. Using spherical coordinates does not exploit the symmetry of the problem
and leads to complicated integration limits, however.