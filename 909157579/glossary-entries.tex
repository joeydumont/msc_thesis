% !TeX root = ../msc_thesis_jayd.tex
\newacronym{sqa}{SQA}{the $\mathbf{S}$- and $\mathbf{Q}$-matrix algorithm}
\newacronym{salt}{SALT}{steady-state \textit{ab initio} laser theory}
\newacronym[\glslongpluralkey={asymmetric resonant cavities}]{arc}{ARC}{asymmetric resonant cavity}
\newacronym[\glslongpluralkey={leaky coax antennae}]{lcx}{LCX}{leaky coax antenna}
\newacronym{fem}{FEM}{finite element method}
\newacronym{fdtd}{FDTD}{finite-difference time-domain}
\newacronym{bem}{BEM}{boundary element method}
\newacronym{bcp}{BCP}{boundary condition problem}
\newacronym{ivp}{IVP}{initial value problem}
\newacronym{vpm}{VPM}{variable phase method}
\newacronym{ode}{ODE}{ordinary differential equation}
\newacronym{pde}{PDE}{partial differential equation}
\newacronym{rwa}{RWA}{rotating wave approximation}
\newacronym{svea}{SVEA}{slowly varying envelope approximation}
\newacronym{sia}{SIA}{stationary inversion approximation}

\newglossaryentry{singValue}%
{%
  name={Singular value and singular eigenvectors},
  description={Related to the eigenvalues and eigenvectors of a matrix...}
}

\newglossaryentry{sMatrix}%
{%
  name={$\mat{S}$-matrix},
  symbol={$\mat{S}$-matrix},
  description={Scattering matrix. In quantum scattering, relates the asymptotes of the
  				incoming field to the asymptotes of the scattered field. For real potentials,
  				this amounts to the phase shifts between them. In electromagnetic scattering,
  				relates the outgoing parts of the field to the incoming parts of it outside 
  				a \textit{last scattering surface} which encloses the whole of the scatterer.
  				It thus contains both near- and far-field information},
  first={scattering matrix},
  plural={scattering matrices}
}

\newglossaryentry{qMatrix}%
{%
  name={$\mat{Q}$-matrix},
  symbol={$\mat{Q}$-matrix},
  description={Time-delay matrix. Related to the scattering matrix. 
  				It describes the energy derivative of the phase
  				shift between the incoming and outgoing field, which
  				can be linked to a time delay due to the interaction
  				with the scatterer},
  first={time-delay matrix},
  plural={time-delay matrices}
}

\newglossaryentry{calPhase}%
{%
  name={calibration phase (algorithms)},
  text={calibration phase},
  plural={calibration phases},
  description={Wherein a numerical algorithm is tested against a problem with a
		known solution. Oftentimes, convergence properties of the algorithm
		are determined using this (usually) trivial scenario}
}

\newglossaryentry{gerschgorin}%
{%
  name={Gerschgorin circles},
  text={Gerschgorin circle},
  plural={Gerschgorin circles},
  description={Gerschgorin's theorem states that the eigenvalues of a matrix $\mat{A}\in\mathbb{C}^{n\times n}$ %
		fall inside the circles whose centers are the diagonal elements of the matrix $\mat{A}$ and whose
		radii are given either the sums of the absolute values of the associated non-diagonal row/column elements,
		whichever is smallest. See main text for details}
}

\newglossaryentry{boundaryCondition}{name={Boundary conditions},description={Conditions on the values of the solution and its normal derivative on specific parts of the domain of definition of the solution. They are necessary to make the PDE well-posed and thus uniquely solvable}}

\newglossaryentry{dirichletBC}%
{%
	name={Dirichlet},
	text={Dirichlet boundary condition},
	description={The value of the solution is specified on the boundary of the domain},
	parent={boundaryCondition}
}

\newglossaryentry{neumannBC}%
{%
	name={Neumann},
	text={Neumann boundary condition},
	description={The normal derivative of the solution is specified on the boundary of the domain},
	parent={boundaryCondition}
}

\newglossaryentry{robinBC}%
{%
	name={Robin}, 
	text={Robin boundary condition},
	description={A linear combination of the solution and its normal derivative is specified on the boundary of the domain},
	parent={boundaryCondition}
}

\newglossaryentry{lss}
{
	name={last scattering surface},
	%text={LSS},
	description={Fictitious surface that encloses the scattering potential. In quantum scattering,
					since the potential usually decreases as $r^{-1-\epsilon}$, it is a
					sphere whose radius tends to infinity. In electromagnetic scattering, 
					the potential is zero outside the cavity region; the LSS is then 
					the smallest sphere that encloses the cavity region. The LSS can be 
					any closed surface, but it is easier to work with spheres, as the differential
					equations usually separate on this surface}
	first={last scattering surface},
	symbol={LSS} 
}
  