% !TeX root = ../msc_thesis_jayd.tex
\chapter{Conclusion}
\section{Summary}
We have presented some results about the modelization of bidimensional
resonators, both passive and active, and have designed a leaky coax
antenna for smart textile applications. 

Throughout this essay, we have made use of the \gls{sMatrix} formalism
to compute the resonances of the devices under study. For passive
(and lossless) cavities, the formalism was augmented by the
introduction of the time-delay matrix, or \gls{qMatrix}, which
is related to the overlap of the normal modes of the resonator.
The eigenvalues of this matrix turned out to be related to the
poles of the \gls{sMatrix} and be composed of a sum of Lorentzian
curves.  The eigenvalues were in fact the time-delays associated with
the interaction of the incoming light field with the scatterer. 
The eigenvectors of the \gls{qMatrix} describe a set of modes that exist for
every continuous value of $k$. The set of modes has the property of 
interacting in a \textit{self-replicating} manner with the cavity. 
While the interaction does not disturb the angular momentum distribution,
it incurs a phase shift on the incoming field, and the energy derivative
of this shift is the time-delay suffered by the mode. It can be directly
related to the $Q$-factor of the mode. The introduction of the 
\gls{qMatrix} transforms the search for the poles of the \gls{sMatrix}
in the complex plane in a search for the maxima of the time-delay spectrum
on the real $k$-line.

The properties of the \gls{qMatrix}, and its simple interpretation, 
mostly depend on the unitarity of the \gls{sMatrix}. When the 
potential is complex, this is no longer true. It is however still
possible to define a Hermitian \gls{qMatrix} with the sought-after properties,
but it requires the evaluation of the \gls{qMatrix} along a curve in 
the complex $k$-plane. This makes the interpretation of the spectrum
of $\mat{Q}$ as a time-delay more difficult, and most importantly 
makes their relationship with the poles of the \gls{sMatrix} clouded.
It also nullifies the computational advantages of working with $k\in\mathbb{R}$.
The author thus concluded that for active (or lossy) cavities, it is best
to stick with the \gls{sMatrix} alone. 

This seemed a perfect segue into the study of active cavities, where
the SALT is used to model the effect of a quantum gain medium as
an additional term in the refractive index\footnote{This is valid for the TM
mode only.}. The latter is dispersive and complex, possibly non-linear.
This motivated the development of a more stable numerical method. 
We used a Lippmann-Schwinger approach. The method is calibrated using
the homogeneous circular cavity laser and we discuss the pole structure
of the scattering matrix.

We then turn our attention to the design of a LCX antenna. The complex
geometry of the antenna and the perplexing features of the metallic
layers made for a complicated modeling task. Some attempts were made
at explaining the major discrepancies between the experimental
and simulation datasets. We quantified the effects of mixing
different materials and of thin films using the Bruggeman
and Fuchs-Sondheimer models, respectively. Using FEM software
to incorporate these models, as well as taking into account
possible surface inhomogeneities via a dedicated module in the
software, the author concluded that an \textit{ab initio}
model of the effect of the surface inhomogeneities was needed
in order to properly assess the thicknesses of the silver layers.

This last paragraph should convince the reader that data alone is 
not sufficient to explain everything. The first section showed that
theory can give both qualitative and quantitative insight into
the systems under study. Data is of course necessary to confirm 
or infirm theory, but, for intellectual and practical reasons, cannot
stand alone. The intellectual exercise of probing the working of the 
universe, in itself, better the state of our knowledge. The resulting
theories provide guidelines for future research and new experiments
\footnote{Guidelines that must sometimes be broken, of course \cite{KUH1996}.}
and ease the interpretation of the data. On the more practical side, the 
guidelines can dramatically reduce the cost of the design phase of 
any kind of device. 

\section{Perspectives}
While most of this essay limited its scope to bidimensional cavities, 
many of the tools can be directly generalized to three dimensions. Much
work was done by the author to find methods that would work reasonably
well in 3D, but it was never seen to completion. The last appendix of this
essay presents some of the basic tools for 3D scattering. They are published
here in the hope that they may be useful in the future. 

One of the main approximations generally used in the study of microcavities
in the assumption that the fields exist in infinitely long cylinders. 
This works reasonably well, but is not enough for accurate comparison with 
experiments. One interesting project would be to use a full 3D numerical
method and investigate the 2D limit of electromagnetism in microcavities, 
if it is well-defined. This could in turn be used to properly model 
polarization effects, which could perhaps be put to use as an 
extra degree of freedom in biodetection devices.

\section{Technical Acknowledgements}
This work would not have been possible without \cite{SAN2010}, a C++ linear algebra
library. We also wish to thank \cite{GEI2013} for the color-blind compliant color maps.