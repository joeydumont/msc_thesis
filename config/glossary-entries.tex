% !TeX root = ../msc_thesis_jayd.tex
\newacronym{sqa}{SQA}{the $\mathbf{S}$- and $\mathbf{Q}$-matrix algorithm}
\newacronym{lss}{LSS}{last scattering surface}
\newacronym{salt}{SALT}{steady-state \textit{ab initio} laser theory}
\newacronym[\glslongpluralkey={asymmetric resonant cavities}]{arc}{ARC}{asymmetric resonant cavity}
\newacronym[\glslongpluralkey={leaky coax antennae}]{lcx}{LCX}{leaky coax antenna}
\newacronym{fem}{FEM}{finite element method}
\newacronym{fdtd}{FDTD}{finite-difference time-domain}
\newacronym{bem}{BEM}{boundary element method}
\newacronym{bcp}{BCP}{boundary condition problem}
\newacronym{ivp}{IVP}{initial value problem}
\newacronym{vpm}{VPM}{variable phase method}
\newacronym{ode}{ODE}{ordinary differential equation}
\newacronym{pde}{PDE}{partial differential equation}
\newacronym{rwa}{RWA}{rotating wave approximation}
\newacronym{svea}{SVEA}{slowly varying envelope approximation}
\newacronym{sia}{SIA}{stationary inversion approximation}

\newglossaryentry{singValue}%
{%
  name={Singular value and singular eigenvectors},
  description={Related to the eigenvalues and eigenvectors of a matrix...}
}

\newglossaryentry{sMatrix}%
{%
  name={$\mat{S}$-matrix},
  symbol={$\mat{S}$-matrix},
  description={Scattering matrix},
  first={scattering matrix},
  plural={scattering matrices}
}

\newglossaryentry{qMatrix}%
{%
  name={$\mat{Q}$-matrix},
  symbol={$\mat{Q}$-matrix},
  description={Time-delay matrix},
  first={time-delay matrix},
  plural={time-delay matrices}
}

\newglossaryentry{calPhase}%
{%
  name={calibration phase (algorithms)},
  text={calibration phase},
  plural={calibration phases},
  description={Wherein a numerical algorithm is tested against a problem with a%
		known solution. Oftentimes, convergence properties of the algorithm
		are determined using this (usually) trivial scenari.}
}

\newglossaryentry{gerschgorin}%
{%
  name={Gerschgorin circles},
  text={Gerschgorin circle},
  plural={Gerschgorin circles},
  description={Gerschgorin's theorem states that the eigenvalues of a matrix $\mat{A}\in\mathbb{C}^{n\times n}$ %
		fall inside the cicles whose centers are the diagonal elements of the matrix $\mat{A}$ and whose
		radii are given either the sums of the absolute values of the associated non-diagonal row/column elements,
		whichever is smallest. See main text for details}
}

\newglossaryentry{boundaryCondition}{name={Boundary conditions},description={Conditions on the values of the solution and its normal derivative on specific parts of the domain of definition of the solution. They are necessary to make the PDE well-posed and thus uniquely solvable}}
\newglossaryentry{dirichletBC}%
{%
	name={Dirichlet},
	text={Dirichlet boundary condition},
	description={The value of the solution is specified on the boundary of the domain},
	parent={boundaryCondition}
}

\newglossaryentry{neumannBC}%
{%
	name={Neumann},
	text={Neumann boundary condition},
	description={The normal derivative of the solution is specified on the boundary of the domain},
	parent={boundaryCondition}
}

\newglossaryentry{robinBC}%
{%
	name={Robin}, 
	text={Robin boundary condition},
	description={A linear combination of the solution and its normal derivative is specified on the boundary of the domain},
	parent={boundaryCondition}
}

  